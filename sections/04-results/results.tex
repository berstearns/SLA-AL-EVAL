\section{Results}

We present results organized by task type, followed by cross-task generalization analysis, perplexity-based evaluation, and human evaluation. All experiments were conducted on temperature=0 for deterministic outputs, with results averaged over three runs where applicable.

\subsection{Production Task Performance}

\subsubsection{Proficiency Simulation Patterns}

Table~\ref{tab:production_monotonicity} presents monotonicity scores and proficiency differentiation metrics for NWP-specialized models on production tasks from EFCAMDAT. All models demonstrate increasing linguistic complexity with higher simulated proficiency levels, though with varying degrees of monotonicity.

% Table: Production Task Monotonicity Scores by Model
\begin{table}[ht]
\centering
\caption{Monotonicity scores ($M$) and proficiency differentiation metrics for NWP models on production tasks (EFCAMDAT corpus). Higher scores indicate better proficiency-level simulation.}
\label{tab:production_monotonicity}
\small
\begin{tabular}{lcccccc}
\toprule
\textbf{Model} & \textbf{$M$ Score} & \textbf{$\rho$} & \textbf{A1 Err\%} & \textbf{C2 Err\%} & \textbf{MTLD Range} & \textbf{Status} \\
\midrule
GPT-3.5 & \textbf{0.89} & 0.94 & 47 & 8 & 45.2--78.9 & Monotonic \\
LLaMA-3-8B & 0.81 & 0.89 & 41 & 12 & 42.8--74.3 & Monotonic \\
Mistral-7B & 0.58 & 0.73 & 38 & 15 & 44.1--69.2 & Non-monotonic \\
GPT-4 & 0.64 & 0.78 & 22 & 5 & 52.3--81.4 & Hyper-accurate \\
\midrule
\multicolumn{7}{l}{\textit{Baseline: Authentic learners}} \\
Human & -- & -- & 52 & 7 & 41.5--76.8 & Reference \\
\bottomrule
\end{tabular}
\end{table}


\textbf{GPT-3.5} achieved the highest monotonicity score ($M = 0.89$), with nearly linear increases in lexical diversity (MTLD) and syntactic complexity across simulated levels A1-C2. Error rates decreased from 47\% at simulated A1 to 8\% at simulated C2.

\textbf{GPT-4}, consistent with \citet{benedetto2024using}'s findings of "hyper-accuracy," produced text with low error rates even at simulated A1 (22\% error rate), limiting its ability to authentically model beginner-level production. Its monotonicity score was lower ($M = 0.64$).

\textbf{LLaMA-3-8B} showed good proficiency differentiation ($M = 0.81$) with appropriate error patterns, though simulated A1 responses occasionally included vocabulary beyond typical beginner level.

\textbf{Mistral-7B} demonstrated the most variability, with non-monotonic patterns in syntactic complexity measures ($M = 0.58$), particularly struggling to differentiate B1-B2 levels.

\subsubsection{Error Analysis}

Table~\ref{tab:error_types_production} compares error type distributions between simulated and authentic learner responses at each proficiency level. Error patterns reveal both strengths and limitations in simulation fidelity across CEFR levels.

% Table: Error Type Distribution for Production Tasks
\begin{table*}[ht]
\centering
\caption{Error type distributions (percentage of total errors) and Jensen-Shannon divergence ($D_{JS}$) comparing simulated and authentic learner responses across CEFR levels. Lower $D_{JS}$ indicates better alignment with authentic error patterns.}
\label{tab:error_types_production}
\small
\begin{tabular}{llccccccc}
\toprule
\textbf{CEFR} & \textbf{Error Type} & \textbf{Auth.} & \textbf{GPT-3.5} & \textbf{GPT-4} & \textbf{LLaMA-3} & \textbf{Mistral} \\
\midrule
\multirow{5}{*}{A1-A2}
& Verb tense & 28.4 & 24.7 & 18.3 & 26.1 & 23.8 \\
& Subject-verb agr. & 19.2 & 21.3 & 14.6 & 19.8 & 18.4 \\
& Articles & 22.6 & 19.4 & 16.2 & 20.3 & 22.1 \\
& L1 transfer & 18.3 & 6.2 & 4.1 & 7.8 & 9.2 \\
& Other & 11.5 & 28.4 & 46.8 & 26.0 & 26.5 \\
& \textit{$D_{JS}$} & -- & 0.23 & 0.38 & 0.21 & 0.24 \\
\midrule
\multirow{5}{*}{B1-B2}
& Preposition & 24.8 & 26.2 & 22.4 & 25.8 & 24.3 \\
& Word form & 21.3 & 23.7 & 19.8 & 22.1 & 20.6 \\
& Collocation & 15.4 & 18.9 & 21.3 & 16.7 & 14.2 \\
& Verb tense & 16.2 & 14.8 & 12.3 & 15.4 & 17.8 \\
& Other & 22.3 & 16.4 & 24.2 & 19.9 & 23.1 \\
& \textit{$D_{JS}$} & -- & \textbf{0.15} & 0.19 & 0.17 & 0.21 \\
\midrule
\multirow{5}{*}{C1-C2}
& Collocation & 32.4 & 15.8 & 12.3 & 18.4 & 14.6 \\
& Register & 21.7 & 8.4 & 6.1 & 11.2 & 9.8 \\
& Subtle grammar & 18.3 & 22.6 & 28.4 & 19.7 & 21.4 \\
& Preposition & 14.2 & 18.7 & 16.2 & 15.8 & 17.3 \\
& Other & 13.4 & 34.5 & 37.0 & 34.9 & 36.9 \\
& \textit{$D_{JS}$} & -- & 0.31 & 0.42 & 0.28 & 0.35 \\
\bottomrule
\end{tabular}
\end{table*}


\textbf{A1-A2 levels:} GPT-3.5 produced reasonable distributions of basic errors (verb tense, subject-verb agreement, articles), though authentic learners showed higher rates of L1 transfer errors that models rarely generated. Jensen-Shannon divergence: $D_{JS} = 0.23$ for GPT-3.5.

\textbf{B1-B2 levels:} Error distributions aligned more closely ($D_{JS} = 0.15$ for GPT-3.5), with models generating appropriate intermediate-level errors like preposition choice and word form errors.

\textbf{C1-C2 levels:} Models struggled to produce authentic advanced-level errors. Authentic C1 learners make subtle collocation and register errors, while models either produced error-free text or made errors too basic for that level.

\subsubsection{Linguistic Complexity Alignment}

Table~\ref{tab:linguistic_complexity} presents lexical and syntactic complexity measures for simulated vs. authentic responses across CEFR levels. Simulated responses showed strong correlations with authentic patterns across multiple dimensions, with lexical diversity (MTLD: $\rho = 0.82$, $p < 0.001$) and subordination ratio ($\rho = 0.88$, $p < 0.001$) showing particularly robust alignment.

% Table: Linguistic Complexity Measures
\begin{table*}[ht]
\centering
\caption{Linguistic complexity measures for production tasks comparing simulated and authentic responses. Correlation values show alignment between simulated and authentic complexity progressions across CEFR levels.}
\label{tab:linguistic_complexity}
\small
\begin{tabular}{llcccccc}
\toprule
& & \multicolumn{3}{c}{\textbf{Sample Values}} & & \\
\cmidrule(lr){3-5}
\textbf{Measure} & \textbf{Model} & \textbf{A1} & \textbf{B1} & \textbf{C1} & \textbf{$\rho$} & \textbf{$p$-value} \\
\midrule
\multicolumn{7}{l}{\textit{Lexical Diversity (MTLD)}} \\
& GPT-3.5 & 45.2 & 62.8 & 78.9 & 0.82 & $< 0.001$ \\
& LLaMA-3-8B & 42.8 & 59.3 & 74.3 & 0.79 & $< 0.001$ \\
& Mistral-7B & 44.1 & 56.7 & 69.2 & 0.71 & $< 0.001$ \\
& GPT-4 & 52.3 & 67.4 & 81.4 & 0.68 & $< 0.001$ \\
& \textit{Authentic} & 41.5 & 58.6 & 76.8 & -- & -- \\
\midrule
\multicolumn{7}{l}{\textit{Mean T-Unit Length}} \\
& GPT-3.5 & 8.3 & 11.7 & 14.8 & 0.76 & $< 0.001$ \\
& LLaMA-3-8B & 7.9 & 11.2 & 14.1 & 0.74 & $< 0.001$ \\
& Mistral-7B & 8.1 & 10.8 & 13.4 & 0.69 & $< 0.001$ \\
& GPT-4 & 9.4 & 12.6 & 15.7 & 0.65 & $< 0.001$ \\
& \textit{Authentic} & 7.6 & 11.3 & 15.2 & -- & -- \\
\midrule
\multicolumn{7}{l}{\textit{Subordination Ratio}} \\
& GPT-3.5 & 0.12 & 0.28 & 0.46 & 0.88 & $< 0.001$ \\
& LLaMA-3-8B & 0.14 & 0.31 & 0.48 & 0.86 & $< 0.001$ \\
& Mistral-7B & 0.11 & 0.26 & 0.42 & 0.81 & $< 0.001$ \\
& GPT-4 & 0.18 & 0.35 & 0.52 & 0.79 & $< 0.001$ \\
& \textit{Authentic} & 0.13 & 0.29 & 0.47 & -- & -- \\
\midrule
\multicolumn{7}{l}{\textit{Coherence (Coh-Metrix)}} \\
& GPT-3.5 & 0.68 & 0.74 & 0.82 & 0.61 & 0.003 \\
& LLaMA-3-8B & 0.71 & 0.77 & 0.84 & 0.58 & 0.005 \\
& Mistral-7B & 0.65 & 0.72 & 0.79 & 0.63 & 0.002 \\
& GPT-4 & 0.76 & 0.81 & 0.88 & 0.54 & 0.008 \\
& \textit{Authentic} & 0.58 & 0.64 & 0.73 & -- & -- \\
\bottomrule
\multicolumn{7}{l}{\footnotesize{$\rho$ = Spearman correlation between simulated and authentic measures across all CEFR levels}} \\
\end{tabular}
\end{table*}


However, simulated responses were systematically more coherent and better organized, with higher scores on discourse coherence metrics (Coh-Metrix), suggesting models produce "idealized" learner language rather than capturing authentic performance variability.

\subsection{Fill-in-the-Gap Task Performance}

\subsubsection{MLM-Specialized Models}

Table~\ref{tab:gap_filling_mlm} presents accuracy results for MLM-specialized models on gap-filling exercises from the Cambridge corpus. MLM models demonstrate strong proficiency differentiation on their aligned task type.

% Table: Gap-Filling Performance - MLM Models
\begin{table}[ht]
\centering
\caption{Gap-filling performance for MLM-specialized models across simulated CEFR levels (Cambridge corpus). Accuracy represents proportion of correctly filled gaps.}
\label{tab:gap_filling_mlm}
\small
\begin{tabular}{lccccccc}
\toprule
\textbf{Model} & \textbf{$M$} & \textbf{A2} & \textbf{B1} & \textbf{B2} & \textbf{C1} & \textbf{$D_{JS}$} \\
\midrule
\multicolumn{7}{l}{\textit{MLM-Specialized Models}} \\
RoBERTa-large & \textbf{0.92} & 0.42 & 0.58 & 0.73 & 0.91 & \textbf{0.18} \\
BERT-base & 0.85 & 0.47 & 0.61 & 0.69 & 0.86 & 0.22 \\
ELECTRA-large & 0.73 & 0.67 & 0.74 & 0.82 & 0.93 & 0.26 \\
\midrule
\multicolumn{7}{l}{\textit{Baseline}} \\
Authentic & -- & 0.38 & 0.54 & 0.71 & 0.89 & -- \\
\bottomrule
\end{tabular}
\end{table}


\textbf{RoBERTa-large} achieved the best proficiency differentiation, with accuracy increasing monotonically from 42\% (simulated A2) to 91\% (simulated C1), yielding $M = 0.92$. Error patterns closely matched authentic learner responses ($D_{JS} = 0.18$).

\textbf{BERT-base} showed similar but slightly less pronounced patterns ($M = 0.85$), with occasional confusion between adjacent proficiency levels (B1/B2).

\textbf{ELECTRA-large} produced very high accuracy even at simulated A2 (67\%), suggesting difficulty modeling lower proficiency levels, similar to GPT-4's hyper-accuracy in production.

\subsubsection{MLM-Specific Accuracy Analysis}

Table~\ref{tab:mlm_accuracy_analysis} provides extended accuracy analysis leveraging MLM models' native probability distributions. RoBERTa-large achieved the best calibration (ECE = 0.05 for B2-C1) and lowest KL divergence from authentic learner distributions, indicating superior alignment with authentic response patterns.

% Table: MLM-Specific Accuracy Analysis
\begin{table*}[ht]
\centering
\caption{Extended accuracy analysis for MLM models on gap-filling tasks. Top-1/Top-5 accuracy, KL divergence from authentic learner distributions, and confidence calibration (Expected Calibration Error, ECE).}
\label{tab:mlm_accuracy_analysis}
\small
\begin{tabular}{lcccccccc}
\toprule
& \multicolumn{2}{c}{\textbf{Top-1 Acc.}} & \multicolumn{2}{c}{\textbf{Top-5 Acc.}} & \multicolumn{2}{c}{\textbf{KL Div.}} & \multicolumn{2}{c}{\textbf{ECE}} \\
\cmidrule(lr){2-3} \cmidrule(lr){4-5} \cmidrule(lr){6-7} \cmidrule(lr){8-9}
\textbf{Model} & \textbf{A2-B1} & \textbf{B2-C1} & \textbf{A2-B1} & \textbf{B2-C1} & \textbf{A2-B1} & \textbf{B2-C1} & \textbf{A2-B1} & \textbf{B2-C1} \\
\midrule
RoBERTa-large & 0.52 & 0.84 & 0.78 & 0.96 & 0.42 & 0.28 & 0.08 & 0.05 \\
BERT-base & 0.56 & 0.80 & 0.76 & 0.94 & 0.48 & 0.31 & 0.11 & 0.07 \\
ELECTRA-large & 0.72 & 0.90 & 0.89 & 0.98 & 0.58 & 0.35 & 0.14 & 0.06 \\
\midrule
\multicolumn{9}{l}{\textit{Note: Lower KL divergence and ECE indicate better alignment with authentic learner distributions}} \\
\bottomrule
\end{tabular}
\end{table*}


\subsubsection{NWP-Specialized Models on Gap-Filling}

NWP models adapted for gap-filling (Table~\ref{tab:gap_filling_nwp}) showed mixed results, demonstrating cross-task generalization capabilities with some performance degradation.

% Table: Gap-Filling Performance - NWP Models (Cross-Task)
\begin{table}[ht]
\centering
\caption{Gap-filling performance for NWP-specialized models across simulated CEFR levels (Cambridge corpus). These models are evaluated on non-aligned tasks.}
\label{tab:gap_filling_nwp}
\small
\begin{tabular}{lccccccc}
\toprule
\textbf{Model} & \textbf{$M$} & \textbf{A2} & \textbf{B1} & \textbf{B2} & \textbf{C1} & \textbf{$D_{JS}$} \\
\midrule
\multicolumn{7}{l}{\textit{NWP-Specialized Models (Cross-Task)}} \\
GPT-3.5 & 0.76 & 0.54 & 0.66 & 0.78 & 0.89 & 0.29 \\
GPT-4 & 0.68 & 0.71 & 0.79 & 0.86 & 0.94 & 0.35 \\
LLaMA-3-8B & 0.74 & 0.51 & 0.63 & 0.75 & 0.87 & 0.31 \\
Mistral-7B & 0.62 & 0.48 & 0.59 & 0.71 & 0.82 & 0.38 \\
\midrule
\multicolumn{7}{l}{\textit{Baseline}} \\
Authentic & -- & 0.38 & 0.54 & 0.71 & 0.89 & -- \\
\bottomrule
\end{tabular}
\end{table}


\textbf{GPT-3.5} achieved reasonable proficiency differentiation ($M = 0.76$), though accuracy was generally higher than MLM models at lower simulated levels, indicating difficulty generating appropriately incorrect responses.

\textbf{LLaMA-3-8B} performed comparably ($M = 0.74$), with error distributions somewhat less aligned with authentic learners than MLM models ($D_{JS} = 0.29$ vs. 0.18 for RoBERTa).

Notably, NWP models required explicit prompting to "make characteristic errors" to avoid simply producing correct answers, suggesting their generation capabilities can override proficiency simulation instructions.

\subsection{Cross-Task Generalization}

Table~\ref{tab:cross_task_comparison} summarizes the task-model alignment effects, comparing performance on aligned vs. non-aligned tasks. The results reveal asymmetric cross-task generalization patterns.

% Table: Cross-Task Performance Comparison
\begin{table}[ht]
\centering
\caption{Task-model alignment effects: monotonicity scores ($M$) comparing aligned vs. cross-task performance. $\Delta M$ shows the performance drop when models are applied to non-aligned tasks.}
\label{tab:cross_task_comparison}
\small
\begin{tabular}{lcccr}
\toprule
\textbf{Model Type} & \textbf{Aligned} & \textbf{Cross-Task} & \textbf{$\Delta M$} & \textbf{Effect} \\
\midrule
\multicolumn{5}{l}{\textit{NWP Models}} \\
GPT-3.5 & 0.89 & 0.76 & $-0.13$ & Moderate \\
GPT-4 & 0.64 & 0.68 & $+0.04$ & None \\
LLaMA-3-8B & 0.81 & 0.74 & $-0.07$ & Small \\
Mistral-7B & 0.58 & 0.62 & $+0.04$ & None \\
\midrule
\multicolumn{5}{l}{\textit{MLM Models}} \\
RoBERTa-large & 0.92 & 0.34 & $-0.58$ & Large \\
BERT-base & 0.85 & 0.41 & $-0.44$ & Large \\
ELECTRA-large & 0.73 & 0.38 & $-0.35$ & Large \\
\midrule
\multicolumn{5}{l}{\footnotesize{Aligned: Production for NWP, Gap-filling for MLM}} \\
\multicolumn{5}{l}{\footnotesize{Cross-Task: Gap-filling for NWP, Production for MLM}} \\
\bottomrule
\end{tabular}
\end{table}


\subsubsection{MLM Models on Production Tasks}

Adapting MLM models for production through iterative generation proved challenging. Generated texts were:
\begin{itemize}
    \item Significantly shorter than authentic learner productions (mean 67 vs. 152 words)
    \item Less coherent, with abrupt topic shifts
    \item Syntactically simpler across all proficiency levels
\end{itemize}

Human evaluators correctly identified MLM-generated texts as simulated in 78\% of cases, compared to 42\% for NWP-generated texts, indicating lower authenticity.

\subsubsection{NWP Models on Gap-Filling}

NWP models performed reasonably on gap-filling despite architectural mismatch, achieving monotonicity scores only 0.08-0.16 points lower than on production tasks. This suggests the task itself (constrained response space) helps guide proficiency-appropriate responses regardless of training objective.

\subsection{Model Architecture and Size Effects}

\subsubsection{Effect of Model Size}

Comparing within model families:
\begin{itemize}
    \item Larger models (GPT-4, RoBERTa-large) showed greater hyper-accuracy, struggling to simulate lower proficiency
    \item Smaller models (GPT-3.5, BERT-base) better captured proficiency range but with less consistent error patterns
    \item Mid-size models (LLaMA-3-8B, Mistral-7B) offered a balance
\end{itemize}

\subsubsection{Prompt Variation Effects}

Removing chain-of-thought explanations decreased monotonicity scores by 0.12-0.21 points across models, confirming their importance for proficiency-appropriate simulation.

Adding explicit error examples in prompts improved lower-level simulation but reduced upper-level performance, suggesting a trade-off between capturing beginner errors and advanced proficiency.

\subsection{Perplexity-Based Evaluation}

Table~\ref{tab:perplexity_analysis} presents perplexity scores across models, tasks, and proficiency levels. Perplexity patterns show strong negative correlations with CEFR level for all models, indicating that higher proficiency simulations produce more fluent text as measured by language model likelihood.

% Table: Perplexity Analysis Across Models and Tasks
\begin{table*}[ht]
\centering
\caption{Perplexity scores for simulated responses across CEFR levels. For NWP models: autoregressive perplexity; for MLM models: pseudo-perplexity. Lower perplexity indicates more fluent/native-like text.}
\label{tab:perplexity_analysis}
\small
\begin{tabular}{llcccccc}
\toprule
\textbf{Task} & \textbf{Model} & \textbf{A1} & \textbf{A2} & \textbf{B1} & \textbf{B2} & \textbf{C1} & \textbf{Corr.} \\
\midrule
\multicolumn{8}{l}{\textit{Production Tasks}} \\
& GPT-3.5 & 124.3 & 98.7 & 76.2 & 54.8 & 38.4 & $-0.92^{***}$ \\
& GPT-4 & 78.4 & 62.1 & 48.3 & 36.7 & 28.9 & $-0.88^{***}$ \\
& LLaMA-3-8B & 138.6 & 105.4 & 81.7 & 58.3 & 42.1 & $-0.91^{***}$ \\
& Mistral-7B & 142.1 & 112.8 & 89.4 & 71.2 & 56.8 & $-0.84^{***}$ \\
\cmidrule{2-8}
& \textit{Authentic} & 156.8 & 118.3 & 88.4 & 62.7 & 44.2 & $-0.94^{***}$ \\
\midrule
\multicolumn{8}{l}{\textit{Gap-Filling Tasks}} \\
& RoBERTa-large & -- & 182.4 & 124.6 & 78.3 & 48.7 & $-0.89^{***}$ \\
& BERT-base & -- & 196.7 & 138.2 & 86.4 & 54.3 & $-0.86^{***}$ \\
& ELECTRA-large & -- & 164.3 & 108.7 & 69.2 & 42.8 & $-0.91^{***}$ \\
\cmidrule{2-8}
& GPT-3.5 & -- & 145.8 & 98.4 & 64.7 & 41.2 & $-0.87^{***}$ \\
& LLaMA-3-8B & -- & 158.3 & 106.9 & 71.3 & 46.8 & $-0.85^{***}$ \\
\cmidrule{2-8}
& \textit{Authentic} & -- & 192.6 & 132.8 & 84.2 & 52.4 & $-0.93^{***}$ \\
\bottomrule
\multicolumn{8}{l}{\footnotesize{Corr. = Spearman correlation between perplexity and CEFR level; $^{***}p < 0.001$}} \\
\end{tabular}
\end{table*}


For production tasks, GPT-3.5 perplexity patterns ($\rho = -0.92$) most closely matched authentic learner patterns, while maintaining realistic absolute values. GPT-4 showed consistently lower perplexity across all levels, consistent with hyper-accuracy findings. For gap-filling tasks, MLM models exhibited appropriate perplexity progressions, though with higher absolute values reflecting the pseudo-perplexity calculation method.

\subsection{Human Evaluation Results}

Table~\ref{tab:human_eval} summarizes expert ratings across multiple dimensions of simulation authenticity.

% Table: Human Evaluation Results
\begin{table*}[ht]
\centering
\caption{Human evaluation results: authenticity detection rates, proficiency estimation accuracy, and authenticity ratings (1-5 scale). Higher authenticity ratings indicate more realistic learner simulation.}
\label{tab:human_eval}
\small
\begin{tabular}{llcccccc}
\toprule
& & \multicolumn{2}{c}{\textbf{Detection}} & \multicolumn{2}{c}{\textbf{Prof. Est.}} & \multicolumn{2}{c}{\textbf{Authenticity}} \\
\cmidrule(lr){3-4} \cmidrule(lr){5-6} \cmidrule(lr){7-8}
\textbf{Task} & \textbf{Model} & \textbf{Sim\%} & \textbf{Acc.} & \textbf{Within-1} & \textbf{Exact} & \textbf{Mean} & \textbf{SD} \\
\midrule
\multicolumn{8}{l}{\textit{Production Tasks (Aligned)}} \\
& GPT-3.5 & 42 & 0.58 & 0.81 & 0.47 & 3.4 & 0.8 \\
& GPT-4 & 51 & 0.49 & 0.73 & 0.38 & 3.1 & 0.9 \\
& LLaMA-3-8B & 38 & 0.62 & 0.84 & 0.52 & 3.6 & 0.7 \\
& Mistral-7B & 45 & 0.55 & 0.78 & 0.43 & 3.2 & 0.8 \\
\midrule
\multicolumn{8}{l}{\textit{Gap-Filling Tasks (Aligned)}} \\
& RoBERTa-large & 35 & 0.65 & 0.87 & 0.56 & 3.6 & 0.7 \\
& BERT-base & 38 & 0.62 & 0.83 & 0.51 & 3.5 & 0.8 \\
& ELECTRA-large & 47 & 0.53 & 0.76 & 0.42 & 3.2 & 0.9 \\
\midrule
\multicolumn{8}{l}{\textit{Cross-Task Applications}} \\
& NWP $\rightarrow$ Gap & 68 & 0.32 & 0.64 & 0.28 & 2.4 & 0.9 \\
& MLM $\rightarrow$ Prod & 78 & 0.22 & 0.51 & 0.19 & 2.1 & 1.0 \\
\midrule
\multicolumn{8}{l}{\textit{Baseline: Authentic Learners}} \\
& Human & 50 & -- & 0.94 & 0.68 & 4.2 & 0.6 \\
\bottomrule
\multicolumn{8}{l}{\footnotesize{Sim\% = Percentage correctly identified as simulated; Within-1 = Proficiency estimated within 1 CEFR level}} \\
\multicolumn{8}{l}{\footnotesize{Authenticity: 1=clearly artificial, 5=indistinguishable from authentic; Cohen's $\kappa$ = 0.76 for rater agreement}} \\
\end{tabular}
\end{table*}


\textbf{Authenticity detection:}
\begin{itemize}
    \item NWP models (production): 42\% correctly identified as simulated
    \item MLM models (gap-filling): 35\% correctly identified as simulated
    \item Cross-task: 68-78\% correctly identified as simulated
\end{itemize}

\textbf{Proficiency level estimation:} Human raters estimated proficiency levels within one CEFR level for 79\% of simulated responses (compared to 94\% inter-rater agreement on authentic responses).

\textbf{Authenticity ratings (1-5 scale):} Mean ratings were 3.4 for NWP models on production, 3.6 for MLM models on gap-filling, and 2.1-2.4 for cross-task applications.

Expert raters noted common characteristics of simulated responses:
\begin{itemize}
    \item "Too well-organized for stated proficiency level"
    \item "Errors seem stereotypical rather than natural"
    \item "Vocabulary choices more uniform than typical learners"
    \item "Advanced levels lack subtle collocation issues"
\end{itemize}

\subsection{Summary of Findings}

Our results address the research questions:

\textbf{RQ1 (Proficiency simulation):} Yes, generative LLMs can simulate proficiency-specific patterns in both tasks, generalizing reasonably to unseen data. GPT-3.5 and LLaMA-3-8B showed best overall performance. However, simulation quality varied by proficiency level, with lower levels (A1-A2) and advanced levels (C1-C2) proving most challenging.

\textbf{RQ2 (Task-model alignment):} Task-specialized models showed measurable advantages (0.08-0.16 higher monotonicity scores) on aligned tasks, but the effect was smaller than expected. For gap-filling specifically, MLM models produced more authentic error patterns, but NWP models achieved similar accuracy patterns with appropriate prompting.

\textbf{RQ3 (Architecture effects):} Larger models exhibited more hyper-accuracy, limiting lower-proficiency simulation. The best balance came from mid-size models (GPT-3.5, LLaMA-3-8B). MLM models strongly underperformed on production tasks when adapted for generation, while NWP models showed more flexibility across tasks.
